自动文本摘要技术(Automatic Text Summarization, ATS)指利用计算模型对冗长源文档进行自动化语义压缩与重构,生成简洁、连贯且保留核心语义的摘要,以解决信息过载问题,同时保持生成摘要的事实性、流畅度和逻辑性。随着大模型时代的到来,自动文本摘要技术已经取的显著进展,但是目前大模型仍存在幻觉现象和上下文长度限制,需要保证模型基于源文档事实生成。因此本文结合了抽取式模型和大语言模型各自的优点,并尝试将两者结合起来进行多轮摘要生成。本文的主要工作和贡献总结如下:



(1)针对文档中语义稀释与位置偏差导致的关健信息遗漏问题,本文基于抽取式摘要模型Bertsum引入双流语义交互机制,首先利用主题模型挖掘文档的显式主题,并通过主题偏置注意力机制将全局主题信息以加性偏置形式注入自注意力层,重塑注意力分布以强化对长距离关键特征的捕捉;另一方面,设计全局门控融合模块,利用文档级主题向量对句子表征进行动态校准,缓解预训练模型固有的位置偏差。此外,通过混合训练目标优化特征空间,驱动模型在潜在语义空间中通过主题锚点对特征分布进行正则化。最后,在多
项标准数据集上对所提出的方法进行了实验,证明了其显著的性能优势。

(2)针对生成式模型存在的幻觉风险与逻辑混乱问题,本文提出了基于思维链驱动的生成-评估-修正闭环框架。首先,通过抽取的关键句构建高信噪比的证据链,并通过局部上下文窗口增强算法修复语义碎片化;随后,引入基于思维链的迭代优化机制:生成器基于证据链生成初稿,评估器通过结构化提示工程对摘要进行事实一致性、语义连贯性、文本逻辑性的检查,并触发预定义操作增补、删除、重写等修正指令;最终,通过多轮迭代使摘要逐步生成最终的摘要。同时,针对静态证据链可能存在信息丢失,本框架还集成反馈驱动的动态检索与语义压缩算法,当评估器检测到信息缺失时,自动发起二次检索并对文本信息进行压缩。


(3)为了将上述的理论研究转化为实际工具,本文设计并开发了一款云端智能文档摘要生成和编辑系统。本系统支持DOCX/PDF等格式的高保真转换,用户可以上传本地文档到系统实现方便的在线编辑和保存,同时系统通过模型上下文协议(MCP)集成了本文提出的算法,支持对文档内容深度问答与摘要生成。一边高效率的阅读科研文献重点内容,也可以对难点内容进行提问并学习新知识,同时在模型的帮助下撰写文档。