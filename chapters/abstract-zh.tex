自然场景图像生成在深度学习的发展中已取得显著进展,但教育领域以及学术领域的图像生成仍然是极大挑战。当前亟需解决的问题主要是:目前商用模型在抽象领域图像生成效果较差,如何利用现有的深度学习方法进行教材概念性文本的生成;因此,本文研究了抽象领域图像生成的方法,主要是从大语言模型的角度以及图像生成模型的角度,并尝试将两者结合起来进行图像生成。本文的主要工作和贡献总结如下:



(1)针对自然场景图像具有文本尺度不一致和形态各异的问题,本文提出了结合尺度估计模块的多路径动态融合的文本检测模型。首先,采用了可变形注意力机制来精细化特征表示;其次,引入了尺度估计模块,对不同感受野的特征图进行重要性评估和权重分配,提高了模型对多尺度文本的检测能力;为了精准定位文本边界框,采用了组合不同级别特征的融合策略,有效提升了文本边界框的定位精度。最后,在多项标准数据集上对所提出的方法进行了实验,证明了其显著的性能优势。

(2)针对遮挡、模糊等低质量图像的识别问题,本文提出了基于视觉语言特征的随机顺序解码的文本识别模型。通过视觉特征精准提取文本特征和位置信息;然后,将初步的视觉信息送入一个基于Transformer的处理模块,以进一步挖掘文本的深层语义特征;在解码阶段,本文提出了一种创新的随机顺序解码规则,使模型能够依据特征质量的高低,选择性地识别字符。最后,本文在六种自然场景文本图像数据集上开展了多组对照实验,以证明本文提出的方法的有效性与优越性。


(3)针对进一步解决任意形状的文本检测与识别的问题,本文提出了基于强化学习的端到端文本检测与识别模型。利用一个特征融合模块整合多层次的特征信息,进而增强检测器对复杂文本布局的理解能力;然后,构建了一个框调整深度Q网络,该网络在文本识别反馈的引导下,能够自动调整包围框的位置和尺寸,确保对密集和任意形状文本的精确检测。此外,还对真值边界框的训练方式进行了优化,以进一步提升端到端场景文本识别模型的性能。最后,通过实验并与当前主流方法进行了对比,结果表明本文所提出的模型在文本检测和识别任务上展现出了显著的性能提升。